\chapter{Protocol of work so far}

\section{Rotated lines experiment}

The goal of this task was to recreate experiment 2 from \citet{nessler}.

Input data:
29 x 29 pixel images of lines going through the center of the image. During the training of the network these lines are generated and randomly oriented for each step.



Excitatory input:
The intrinsic weights of the x neurons were omitted in this experiment. This was chosen because there was no benefit in including them.
\begin{equation}
\label{eqn:uk}
U_k(t) = \sum_{i=1}^n w_{ki} \cdot x_i(t)
\end{equation}

Firing rate:
\begin{equation}
\label{eqn:rk}
r_k(t) = e^{u_k(t) - I(t)}
\end{equation}

Chance to spike within time step \delta t:
\begin{equation}
\label{eqn:rkdt}
r_k(t) \cdot \delta t
\end{equation}

Weight updates:
\begin{equation}
\label{deltawki}
\Delta w_{ki} = \begin{dcases*} ce^{-w_{ki}} - 1 & if $x_{i}(t^f) = 1$, i.e.$x_{i}$ fired in $ [t^f - \sigma, t^f] $ \\
-1 & \text{if $ y_i(t^f) = 0 $, i.e. $ y_i $ did not fire in $ [t^f - \sigma, t^f] $ } \end{dcases*}
\end{equation}