\section{Winner-Take-All networks}


1. WTA networks

\cite softWTA
Figure 6: pyramidal neurons on layers 2/3 and 5/6 typically inhibit each other => Those are called soft winner-take-all WTA circuits.

ORIGINAL:

A simple model of cortical processing, consistent with the major features of cortical
circuits discussed in this review, is as follows (see Figure 6): A patch of superficial
pyramidal neurons receive feedforward excitatory input from subcortical, interareal, and intra-areal sources. In addition to their interactions with their close
neighbors within their patch, the members of this patch also receive feedback
from a number of sources: from deep pyramidal cells immediately beneath their
patch, from other close patches within the superficial layers, and from subcortical
inter-areal connections. Thus, the neurons of a superficial patch, taken as a group,
receive a sample of thalamic input (some preprocessed by layer 4), a sample of
surrounding and remote superficial patches, and a sample of the output from their
corresponding deep pyramidal neurons.
All of these inputs are processed by the dendrites of the superficial pyramids
whose signal transfer properties can be adjusted dynamically by the pattern of
the vertical inputs from smooth cells (e.g., double bouquet cells). The superficial pyramids collectively participate in a selection network, mediated by the horizontal
inputs from the smooth cells that control their outputs (e.g., basket and chandelier
cells). 

https://en.wikipedia.org/wiki/Basket_cell  ... inhibieren in dem sie ein axon zum soma einer anderen zelle verbinden und das axon hat dann viele dendriten (?) ums soma der anderen zelle und discharched diese dadurch.

das neuron lesen auf wikipedia...

gehrin/cortex überfliegen...

https://en.wikipedia.org/wiki/Chandelier_cell
... have synapses that connect to exclusively to initial axon segment of pyramidal cells and inhibiting them.	

https://en.wikipedia.org/wiki/Pyramidal_cell
.. primary excitation units in prefrontal cortex

The selection mechanism is a soft winner-take-all or soft MAX mechanism, which are important elements of many neuronal network models (Maass
2000, Riesenhuber & Poggio 1999, Yuille & Geiger 2003). The outputs of the
selected superficial pyramids feed back to adapt the pattern of vertical smooth cell
activation. In this way, the superficial layer neurons within and between patches,
and within and between areas, cooperate to explore all possible interpretations
of input, and so select an interpretation consistent with their various subcortical
inputs.
The superficial layers are organized to distribute and explore possible interpretations, whereas the deeper layers are organized to exploit the evolving interpretations. The pyramidal cells of layer 5 that drive subcortical structures
involved in action (e.g., basal ganglia, colliculus, ventral spinal cord) decide the
output of the cortical circuits. The same layer 5 pyramidal cells influence the ongoing input by their connection to layer 6 pyramidal cells that connect to the thalamic
input layers. The explorative processing in the superficial layers is constrained via
the recurrent projection from other layer 5 pyramidal cells to conform to the output
that has already been decided. These layer 5 pyramidal cells are also the origin of
the feedback projections to the superficial layers of other cortical areas. In this way,
they also provide additional contextual information to the evolving interpretations
occurring in the superficial layers of other cortical areas.
Clearly, this model is a tentative hypothesis of how the generic circuits might
express themselves functionally. However, its strength is that it casts antomical
data in a way that is accessible to theoreticians and systems physiologists. The
investigation of neocortical structure and its development has entered an exciting
phase in which the detailed organization is accessible to experiment and essential
to the theoretical understanding of cortical computation. It is thus a curious paradox that while molecular biology has long recognized the central importance of
detailed structural studies for understanding function, the same cannot be said for
contemporary neuroscience.

FIG 6:
Simple model of cortical processing incorporating the principal features of cortical
circuits. A patch of superficial pyramidal neurons receive feedforward input from subcortical, inter-areal, and intra-areal excitatory sources. They also receive recurrent input from
other local superficial and deep pyramidal cells. These inputs are processed by dendrites of
the superficial pyramidal neurons (upper gray rectangles, layer 2/3) whose signal transfer
properties are adjusted dynamically by the pattern of vertical smooth cell inputs (oblique
dark gray arrows). The outputs of the superficial pyramids participate in a selection network
(e.g., soft winner-take-all mechanism) mediated by the horizontal smooth cells (upper horizontal dark gray line). These outputs of the superficial pyramids adjust the pattern of vertical
smooth cell activation. In this way, the superficial layer neurons within and between patches,
and within and between areas, cooperate to resolve a consistent interpretation. The layer 5
pyramids (lower gray rectangles) have a similar soft selection configuration (lower dark gray
line) to process local superficial signals and decide on the output to motor structures.


\cite WTAPower


\cite handbookWTA


\section{The hierarchical structure of the brain}

The brain is made up of many modular structures that connected in a hierarchical organization. These modules have a high level of connectivity within themselves and a low level to other modules. This means that there are different categories of neurons within a module, depending on their function in the network. For one there are provincial hub neurons that primarily connect to other neurons of the same module and are responsible for the function the module expresses. Then there are connector hub neurons that transfer information from the module to other modules. Such modules form networks which are connected in a hierarchical manner, where each module is adding to the output of the previous module. \citep{hierarchicalBrain}
The most of the information flows upwards along the hierarchy, but there is physiological experimental data that show that information is also passed downwards and influences the activity of modules lower in the hierarchy. This will be explained for the visual cortex.

\section{Visual cortex}

wikipedia visual cortex (https://en.wikipedia.org/wiki/Visual_cortex
https://en.wikipedia.org/wiki/Lateral_geniculate_nucleus):
 Sensory input originating from the eyes travels through the lateral geniculate nucleus in the thalamus and then reaches the visual cortex. The area of the visual cortex that receives the sensory input from the lateral geniculate nucleus is the primary visual cortex, also known as visual area 1 (V1), Brodmann area 17, or the striate cortex. The extrastriate areas consist of visual areas 2, 3, 4, and 5 (also known as V2, V3, V4, and V5, or Brodmann area 18 and all Brodmann area 19)
 Studies involving blindsight have suggested that projections from the LGN travel not only to the primary visual cortex but also to higher cortical areas V2 and V3. Patients with blindsight are phenomenally blind in certain areas of the visual field corresponding to a contralateral lesion in the primary visual cortex; however, these patients are able to perform certain motor tasks accurately in their blind field, such as grasping. This suggests that neurons travel from the LGN to both the primary visual cortex and higher cortex regions.[14]
The axons that leave the LGN go to V1 visual cortex
The LGN receives information directly from the ascending retinal ganglion cells via the optic tract and from the reticular activating system. Neurons of the LGN send their axons through the optic radiation, a direct pathway to the primary visual cortex. In addition, the LGN receives many strong feedback connections from the primary visual cortex.[1]
 Neurons in the visual cortex fire action potentials when visual stimuli appear within their receptive field.
By definition, the receptive field is the region within the entire visual field that elicits an action potential. But, for any given neuron, it may respond best to a subset of stimuli within its receptive field. This property is called neuronal tuning. In the earlier visual areas, neurons have simpler tuning. For example, a neuron in V1 may fire to any vertical stimulus in its receptive field. In the higher visual areas, neurons have complex tuning. For example, in the inferior temporal cortex (IT), a neuron may fire only when a certain face appears in its receptive field.
V1 macht edge detection und schickt zu V2
V1 stellt quasi ein bild dar, jedes neuron schaut auf einen bereich des sichtfeldes, das daneben auf den bereich daneben
V2 schaut auf texture, depth und color
auge => LGN => V1 => V2 => V3,4,5
das ist die hierarchical structure, jeder fügt was hinzu
 reciprocal (rückführende) connections of V2 to V1 modulate activity of V1 (mein paper)

3. LeeTS explain experiments and results of feedback information of higher visual areas

4. high degree of uncertainty in visual input... face in shadow. (LeeTS)

5. Several experiments on animals show that the cerebral
cortex utilizes Bayesian inference to compute information like sensory
input (Funamizu Akihiro, 2016; Lee TS, 2003; Parr and Friston, 2018). They
hypothesize that the brain functions as a generative (probabilistic) model
to reach its conclusions. 

\section{Spike timing-dependent plasticity}

6. STDP plasticity, LTP, LTD