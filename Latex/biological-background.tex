\section{Winner-Take-All networks}

It has been found that pyramidal neurons on layers 2/3 and 5/6 of the neocortex form selection networks. This means that one pyramidal neuron is allowed to fire, while others within the same layer are inhibited. This horizontal selection mechanism determines what is vertically sent on to deeper layers of the cortex. For example pyramidal neurons in layer 5 have such a selection behaviour to determine the output to motor structures. This inhibition is performed by basket and chandelier cells \citep{softWTA}. Basket cells connect to the soma of other neurons and control the action potential discharge rate of those neurons. Their dendritic branches wrap around the target soma, forming a "basket" giving them their name \citep{basketCells}. Chandelier cells perform inhibition directly at the axonal initial segment of pyramidal neurons where action potentials are initiated. \citep{chandelierCells}. The selection of one pyramidal neuron performed this way is called soft winner-take-all mechanism and is used in various neuronal network models \citep{softWTA}. The mechanism is called "soft" because it uses a softmax function, which controls how many winners there can be, and how likely each neuron is to win \citep{ handbookWTA}. It has been shown that winner-take-all mechanisms are computationally powerful compared to threshold gates (McCulloch-Pitts neurons) and sigmoidal gates. It was also possible to approximate arbitrary continuous function by circuits that only contained one soft winner-take-all gate as their only nonlinear operator \cite WTAPower.

\section{Spike timing-dependent plasticity}

6. STDP plasticity, LTP, LTD

\section{The hierarchical structure of the brain}

The brain is made up of modular structures that connect with each other in a hierarchical organization. These modules have a high level of connectivity within themselves and a low level to other modules. This means that there are different categories of neurons within a module, depending on their function in the network. For one there are provincial hub neurons that primarily connect to other neurons of the same module and are responsible for the function the module expresses. Then there are connector hub neurons that transfer information from the module to other modules. Such modules form networks which are connected in a hierarchical manner, where each module is adding to the output of the previous module. \citep{hierarchicalBrain}
Most of the information flows upwards along the hierarchy, representing bottom-up observations. But there is physiological experimental data that shows that information is also passed downwards and influences the activity of modules lower in the hierarchy, representing top-down context. This will be explained for the visual cortex.

\section{Visual cortex}

The visual cortex is the region of the brain that processes visual information coming from the retina. From the retina the information is sent to the lateral geniculate nucleus in the thalamus and then further to the primary visual cortex, also called visual area 1 (V1). The visual cortex consists of five visual areas (V1 to V5), which are divided by their function and structure. These areas are located in the occipital lobe of the cerebral cortex. The purpose of the visual areas is to process visual information to recognize objects, perform spatial tasks or to perform  visual-motor skills. Neurons of the visual cortex often respond to stimuli within a specific receptive field. It is assumed that each subsequent visual area is more specialized than the previous and due to that neurons in different visual areas respond to the same receptive fields, but to different types of stimuli. There are various specialized cells in the visual cortex. For example simple cells and complex cells are well studied. Simple cells which mainly occur in V1 respond primarily to oriented edges and lines within a receptive field. For example a simple cell would always fire action potentials when there is a horizontal line in its receptive field. Complex cells can be found in V1, V2 and V3 and also respond primarily to oriented edges and lines. However their receptive fields are larger and an horizontal edge for example does not have to be at a specific location in the receptive field to activate the cell. Some complex cells even respond primarily to movement of edges. \citep{visualCortexBook}
This activation of neurons depending on the stimulus is called neuronal tuning. The stimuli the complex cells respond to get more complex, the higher up in the visual cortex hierarchy they are located. For example in the inferior temporal cortex, which receives visual information from V4, there are complex cells that respond to faces. \citep{complexCellsFaces}
According to \citet{complexCellsIntegrated} the larger receptive fields of complex cells are due to the hierarchical convergent nature of visual processing. This nature follows from the complex cell receiving input from many simple cells which is summed and integrated.

Visual information first enters V1, which is the best-understood part of the visual cortex. It consists of six layers that function differently. Layer 4 receives the input from the lateral geniculate nucleus. It has the most simple cells of the six layers and thus processes visual information of small receptive fields. On layers 2, 3 and 6 there are complex cells, which combine the result of layer 4 into larger receptive fields. Through that V1 outputs simple visual components with their orientation or direction. The processed information is the sent on to V2 which further processes it and thus responds to more visual complex patterns. V2 was also found to respond to differences in color and spatial frequency, additional to the more complex patterns and object orientation. After processing V2 sends its information on to V3, V4 and V5. Furthermore it also has feedback connections to V1, which will be talked about later. After V2 the visual information is split up into the dorsal and ventral streams, which each specialize in processing different features of the visual information. The dorsal stream is involved in guidance of motor actions and in recognizing where objects are in space. The ventral stream on the other hand is responsible for object recognition and form representation.
\citep{visualCortexBook}

LeeTS explain experiments and results of feedback information of higher visual areas
high degree of uncertainty in visual input... face in shadow.

\section{Probabilistic brain}
Several experiments on animals show that the cerebral
cortex utilizes Bayesian inference to compute information like sensory input.

Citep einfach mit komma drinnen trennen.

Funamizu Akihiro, 2016 citep{neuralSubstrate}

Lee TS, 2003 citep{leeTS} 
They emphasize that inference is more general than just competition and that feedback should not only be conceived in terms of attentional selection or biased competition, but rather that it also influences the inference and processing. (eg show a line where there is non, rather than selecting 1 of 2 existing ones.

Parr and Friston, 2018 citep{anatomyOfInference} 

\citep{neuralImplementationOfBayesionInferenceSensoryMotor}
Actions are guided by a Bayesian-like interaction between priors based on experience and current sensory evidence. Here, we unveil a complete neural implementation of Bayesian-like behavior, including adaptation of a prior. 


They hypothesize that the brain functions as a generative (probabilistic) model to reach its conclusions. 