\section{Winner-Take-All networks}


1. WTA networks
Nessler: [16] pyramidal neurons on layers 2/3 and 5/6 typically inhibit each other => Those are called soft winner-take-all WTA circuits

\section{The hierarchical structure of the brain}

The brain is made up of many modular structures that connected in a hierarchical organization. These modules have a high level of connectivity within themselves and a low level to other modules. This means that there are different categories of neurons within a module, depending on their function in the network. For one there are provincial hub neurons that primarily connect to other neurons of the same module and are responsible for the function the module expresses. Then there are connector hub neurons that transfer information from the module to other modules. Such modules form networks which are connected in a hierarchical manner, where each module is adding to the output of the previous module. \citep{hierarchicalBrain}
The most of the information flows upwards along the hierarchy, but there is physiological experimental data that show that information is also passed downwards and influences the activity of modules lower in the hierarchy. This will be explained for the visual cortex.

\section{Visual cortex}

wikipedia visual cortex (https://en.wikipedia.org/wiki/Visual_cortex
https://en.wikipedia.org/wiki/Lateral_geniculate_nucleus):
 Sensory input originating from the eyes travels through the lateral geniculate nucleus in the thalamus and then reaches the visual cortex. The area of the visual cortex that receives the sensory input from the lateral geniculate nucleus is the primary visual cortex, also known as visual area 1 (V1), Brodmann area 17, or the striate cortex. The extrastriate areas consist of visual areas 2, 3, 4, and 5 (also known as V2, V3, V4, and V5, or Brodmann area 18 and all Brodmann area 19)
 Studies involving blindsight have suggested that projections from the LGN travel not only to the primary visual cortex but also to higher cortical areas V2 and V3. Patients with blindsight are phenomenally blind in certain areas of the visual field corresponding to a contralateral lesion in the primary visual cortex; however, these patients are able to perform certain motor tasks accurately in their blind field, such as grasping. This suggests that neurons travel from the LGN to both the primary visual cortex and higher cortex regions.[14]
The axons that leave the LGN go to V1 visual cortex
The LGN receives information directly from the ascending retinal ganglion cells via the optic tract and from the reticular activating system. Neurons of the LGN send their axons through the optic radiation, a direct pathway to the primary visual cortex. In addition, the LGN receives many strong feedback connections from the primary visual cortex.[1]
 Neurons in the visual cortex fire action potentials when visual stimuli appear within their receptive field.
By definition, the receptive field is the region within the entire visual field that elicits an action potential. But, for any given neuron, it may respond best to a subset of stimuli within its receptive field. This property is called neuronal tuning. In the earlier visual areas, neurons have simpler tuning. For example, a neuron in V1 may fire to any vertical stimulus in its receptive field. In the higher visual areas, neurons have complex tuning. For example, in the inferior temporal cortex (IT), a neuron may fire only when a certain face appears in its receptive field.
V1 macht edge detection und schickt zu V2
V1 stellt quasi ein bild dar, jedes neuron schaut auf einen bereich des sichtfeldes, das daneben auf den bereich daneben
V2 schaut auf texture, depth und color
auge => LGN => V1 => V2 => V3,4,5
das ist die hierarchical structure, jeder fügt was hinzu
 reciprocal (rückführende) connections of V2 to V1 modulate activity of V1 (mein paper)

3. LeeTS explain experiments and results of feedback information of higher visual areas

4. high degree of uncertainty in visual input... face in shadow. (LeeTS)

5. Several experiments on animals show that the cerebral
cortex utilizes Bayesian inference to compute information like sensory
input (Funamizu Akihiro, 2016; Lee TS, 2003; Parr and Friston, 2018). They
hypothesize that the brain functions as a generative (probabilistic) model
to reach its conclusions. 

\section{Spike timing-dependent plasticity}

6. STDP plasticity, LTP, LTD