%%%% Time-stamp: <2013-02-25 10:31:01 vk>
\section*{Kurzfassung}
\label{cha:abstractGerman}

Diese Arbeit untersucht hierarchische Architekturen von spikenden Winner-Take-All (WTA) Netzwerken, mit dem Fokus auf der Simulation von Rückkopplungsmechanismen, welche im visuellen Kortex gefunden wurden. Solche Rückkopplungsmechanismen hängen mit Aufmerksamkeit und konsistenten Überzeugungen, über verschiedene Areale des visuellen Kortex hinweg, zusammen.
Zunächst werden die biologischen und theoretischen Konzepte von spikenden neuronalen Netzwerken und ihre Beziehung zur Bayesschen Inferenz gezeigt. Es wird verdeutlicht, wie die hypothetisierte probabilistische Natur des Gehirns mit dem Modell eines spikenden WTA Netzwerks, welches Bayessche Inferenz durchführt, verknüpft werden kann.
Die Arbeit umfasst eine Reihe von Experimenten, die entworfen wurden, um die Reaktion des Netzwerks auf visuelle Reize zu testen. Die Reaktion des Netzwerks auf mehrdeutige Bilder wird getestet und es wird gezeigt, dass das hinzugefügte Feedback einen entscheidenden Beitrag zur Interpretation solcher Bilder leistet. Der Effekt des visuellen Kortex, illusorische Linien zu sehen, wird reproduziert, indem dem Netzwerk Feedback gegeben wird, das den visuellen Reizen widerspricht. Außerdem wird der Zusammenhang zur Bayesschen Inferenz überprüft, indem die bedingten Wahrscheinlichkeiten von Neuronen berechnet werden und ihre synaptischen Gewichte von ihnen abgeleitet werden. Um ein besseres Verständnis des Netzwerkmodells zu gewinnen, wird der Einfluss von Hyperparametern des Netzwerks analysiert. Eine unerwartete Eigenschaft der Feuerraten der Eingangs- und Priorneuronen wird gezeigt, welche die Wahrscheinlichkeitsverteilung des Ausgangs des Netzwerks verändert. Die Wiederverwendbarkeit von Hyperparametern von Netzwerken unterschiedlicher Größen wird getestet. Es wird gezeigt, dass die Hyperparameter nicht universell auf alle Netzwerkgrößen anwendbar sind. Die Qualität des Trainings des Netzwerks wurde mit dem analytischen Optimum verglichen und es wurde gezeigt, dass das Training es nicht erreichen konnte. Diese Experimente in Verbindung zeigen, dass die Einbindung von Feedback die Fähigkeit des Netzwerks visuelle Informationen zu verarbeiten verbessert und Verhaltensmuster widerspiegelt, die in biologischen Systemen beobachtet wurden.

%\glsresetall %% all glossary entries should be used in long form (again)
%% vim:foldmethod=expr
%% vim:fde=getline(v\:lnum)=~'^%%%%\ .\\+'?'>1'\:'='
%%% Local Variables:
%%% mode: latex
%%% mode: auto-fill
%%% mode: flyspell
%%% eval: (ispell-change-dictionary "en_US")
%%% TeX-master: "main"
%%% End:
