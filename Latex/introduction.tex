Several experiments on animals show that the cerebral cortex utilizes Bayesian inference to compute information like sensory input \citep{neuralSubstrate, leeTS, anatomyOfInference}. These experiments hypothesize that the brain functions as a generative (predictive) model to reach its conclusions. This means, rather than utilizing static neural codes to represent information, the brain expresses information via probability distributions. In the context of cortical computation Bayes theorem yields a posterior probability (the prediction), by multiplying the likelihood (sensory input) with the prior probability (additional information not contained in the sensory input). Lee TS and Mumford D. present experimental evidence that information is passed backwards from high-level areas to low-level areas of the visual cortex. This information can be interpreted as the prior probability of Bayes theorem. 



leeTS has evidence for hierarchical baysian inference in visual cortex.

nessler und nesslerClone created winner take all spiking neural network based on bayesian inference. 


This thesis aims to establish a link between the biological evidence of hierarchical cortical computation of the brain and computer simulations of neural networks. 