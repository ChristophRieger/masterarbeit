%%%% Time-stamp: <2013-02-25 10:31:01 vk>

\section*{Abstract}
\label{cha:abstract}

This thesis explores hierarchical architectures of spiking Winner-Take-All (WTA) networks, with the focus on simulating feedback mechanisms found in the visual cortex. Such feedback mechanisms are related to attention and consistent beliefs across different areas of the visual cortex. 
First, the biological and theoretical concepts of spiking neural networks and their relationship to Bayesian inference are established. It is demonstrated how the hypothesized probabilistic nature of the brain can be linked to the model of a spiking WTA network that performs Bayesian inference.
The thesis involves a series of experiments designed to test the network's response to visual stimuli. The response of the network to ambiguous images is tested and it is shown that the added feedback makes a crucial contribution to interpreting such images. An effect of the visual cortex of seeing illusory lines is reproduced by feeding the network feedback that contradicts the visual input. Furthermore, the model's link to Bayesian inference is verified by calculating conditional probabilities of neurons and then deriving their synaptic weights from them. To gain a better understanding of the network model the impact of the different network hyperparameters is analysed. An unexpected property of the firing frequencies of input and prior neurons is discovered, which changes the output's probability distribution of the network. The reusability of hyperparameters to networks of different sizes is tested. It is found that the hyperparameters are not universal to all network sizes. The quality of the training of the network was compared to the analytical optimum and it was found that the training could not reach it. These experiments combined reveal that incorporating feedback enhances the network's ability to process visual information, reflecting behaviours observed in biological systems.

%\glsresetall %% all glossary entries should be used in long form (again)
%% vim:foldmethod=expr
%% vim:fde=getline(v\:lnum)=~'^%%%%\ .\\+'?'>1'\:'='
%%% Local Variables:
%%% mode: latex
%%% mode: auto-fill
%%% mode: flyspell
%%% eval: (ispell-change-dictionary "en_US")
%%% TeX-master: "main"
%%% End:
