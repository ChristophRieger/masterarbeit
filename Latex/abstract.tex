%%%% Time-stamp: <2013-02-25 10:31:01 vk>


\chapter*{Abstract}
\label{cha:abstract}

This thesis explores hierarchical architectures of spiking Winner-Take-All (WTA) networks, with the focus on simulating feedback mechanisms found in the visual cortex. While traditional models of the visual pathway have predominantly followed a feed-forward approach recent neurophysiological evidence was found that indicates the presence of significant feedback loops. These feedback mechanisms are related to attention and consistent beliefs across different of the visual cortex. 

The research begins by expanding previous spiking WTA network models by incorporating hierarchical feedback. After that  the biological and theoretical concepts of spiking neural networks and their relationship to Bayesian inference are established. It is demonstrated how the hypothesized probabilistic nature of the brain can be linked to the model of a spiking WTA network that performs Bayesian inference.

The thesis involves a series of experiments designed to test the network's response to visual stimuli. The response of the network to ambiguous images is tested and it is shown that the added feedback makes a crucial contribution to interpreting such images. The observed behaviour of the network when shown ambiguous figures is interpreted as an attention mechanism. Furthermore, its link to Bayesian inference is verified by calculating conditional probabilities of neurons and then deriving their synaptic weights from them. When setting the feedback to contradict the visual input the network shows the same behaviour as observed in the visual cortex, where it indicates to see something, even if no visual input is present to support that claim. To gain a better understanding of the network model the impact of the different network hyperparameters is analysed. An unexpected property of the firing frequencies of input and prior neurons is discovered, which changes the output probability distribution of the network. The reusability of hyperparameters to networks of different sizes is tested. It is found that the hyperparameters are not universal to all network sizes, due to the way the network is modelled. The quality of the training of the network was compared to the analytical optimum and it was found that the training could not quite reach it. At least one more hyperparameter would be needed to further improve the training process. These experiments combined reveal that incorporating feedback enhances the network's ability to process visual information, reflecting behaviours observed in biological systems.

In conclusion, this thesis provides a step towards more biologically plausible neural network models by demonstrating the significance of feedback information. A better understanding of the spiking WTA network model is fostered and ideas to further expand and improve this model are provided.

%\glsresetall %% all glossary entries should be used in long form (again)
%% vim:foldmethod=expr
%% vim:fde=getline(v\:lnum)=~'^%%%%\ .\\+'?'>1'\:'='
%%% Local Variables:
%%% mode: latex
%%% mode: auto-fill
%%% mode: flyspell
%%% eval: (ispell-change-dictionary "en_US")
%%% TeX-master: "main"
%%% End:
